%#####################################################################
\chapter{まとめ・今後の課題}
%#####################################################################

 本章では,本研究のまとめを行った後,今後の課題を述べる.

本研究は,スーパーコアに関して考案した新たなネットワークモデルの構築をネットワークシミュレータns-3上で行った.
当初の目標である,OpenFlowを用いてのスーパーコアを経由したパケット転送の制御は,検証結果を見てみると全て意図したアルゴリズムの通り問題なく実装されたと考えられる.
これにより,従来のスイッチのように複雑なテーブルを全て手動で制御することなく,大部分をコントローラに任せることができる.
更に,本研究で用いたネットワークモデルを用いることで,現在スーパーコアで使用されるIPSだけでなく,様々なIPSを用いて従来のような性能が期待できると考えられる.

本研究ではOpenFlowを用いたモデル構築の初期段階しか扱うことができなかったが,今後OpenFlowを用いた大規模なモデル構築をする際の一助となればよいと考える.
更に,ns-3を用いてネットワークモデルを構築するという内容に関する日本語論文およびテキスト,Webページが滅多に存在せず,更にOpenFlowを用いるという条件を加えるとその数は更に減少する.
本論文により,OpenFlowとns-3を用いたシミュレーションの一例が増えることに貢献できたと考える.
% 今後,技術が更に発展しOpenFlowが実用化に近づくことを切に願うばかりである.

最後に,本研究を行った上での今後の課題を以下に示す.

一つ目の課題として,構築したネットワークモデルと現EUNETとの詳細な仕様の乖離が挙げられる.
本研究では,通信されるパケットは全てスーパーコアを経由するという前提のもとで,ネットワークモデルを構築した.
しかし,現EUNETはVLANなどを用いてセグメントを分けるという処理を行っている.
ホスト間のセグメントが別ならば,通信されるパケットはスーパーコアを用いて検疫した後,宛先のホストへと届けられるが,セグメントが同じならば,通常のスイッチおよびルータと同じように振る舞い,パケットを通信する.
つまり,OpenFlow技術を用いてEUNETを詳細に表現するには,以上のような仕様追加を行い,再度シミュレーションをする必要がある.

二つ目の課題は,OpenFlowスイッチ,コントローラおよびホストいずれも実機を用いた実験を行っていないということである.
2.1節で説明したように,ns-3ではTAPデバイスを用いることによって,シミュレーションモデルの一部を実機に繋ぐことができ,より現実的な実験を行うことができる.
実際にEUNETをOpenFlowを用いて構築しようとするとき,一般に市販されているOpenFlow対応スイッチおよびコントローラを用いることがほとんどだと考えられる.
しかし,今回の実験では市販品を用いた実験ではなく,全てns-3および別添のOpenFlowモジュール群でサポートされているモジュールを用いてスイッチおよびコントローラを作成した.
つまり,本研究で提案した手法はそのまま引き継げる可能性があるが,作成したプログラムをそのまま適用できないことが考えられる.
更に,本研究はOpenFlow 0.8.9というOpenFlowの中で一番古いバージョンを用いている.
OpenFlowの制約も,導入するときのバージョンによって異なる可能性があるため,今回の手法が現在のOpenFlow技術に対して有効的でない可能性があると考えられる.