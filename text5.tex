%#####################################################################
\chapter{まとめ・今後の課題}
%#####################################################################

% 森定さんの内容
\begin{comment}
 本研究では,ネットワークシミュレータns-3 上でシミュレーションモデルの構築をおこなってきた.
なお,ns-3に関してC++で開発を行うことに関する日本語文献,大規模ネットワークの構築事例がこれまで存在せず,ソースコードも非常に僅かであることから,本研究はns-3上でのネットワーク構築マニュアルとして強く貢献することができる.
本論文のネットワーク構築法に基づいて,ns-3 上でシナリオを記述することで,意図した様々なネットワークを無料で構築することができる.
これにより,ns-3 上で構築した様々なネットワークの挙動確認や性能評価による改善点の発見を行えるだけでなく,本研究で使用した汎用的なモデルを利用し,新たなネットワークの性能評価を行う際のモデル化方法や,実装内容の指標とすることができる.

今後の課題は以下の通りである.
まず第一に挙げられるのが,ルータモジュールの完成である.
現在,ルータモジュールを表現するクラスとしてSimpleRouter,EunetRouterといったクラスを開発したが,現在はDCEによりQuaggaデーモンを利用して自動的にルーチング処理を行うようなルータしか開発できていない.
EUNETをモデル化するためにはVLANを設置しその設定を行うための機能追加が望まれる.
ルーチングアルゴリズムについても独自の変更を加えたアルゴリズムを用意する必要がある.EUNETではVLANを超えて送受信されるパケットを,一度最上位であるスーパーコアで検疫し,再度EUNET上に送信するような経路選択を行うよう設計しているためである.\\

次に,作成したモジュールを現在テストケースとして用いているEUNETのトポロジ記述クラスに反映させ,人間が把握できる程度の少ないパケットを流して解析し,大規模ネットワークにおいても作成したL3スイッチモジュールが正常に動作するのかの確認を行う必要がある.
EUNET独自のルーチングテーブルに従い,パケット伝送経路をたどるかパケットキャプチャを用いて解析し,評価を行う必要がある.

最後に,シミュレーションを行う場合に必要な項目としてパラメータを決定する基準を準備する必要がある.
EUNETに使用される機器についてのパラメータは,仕様書によりそのパラメータを設定する事ができるが,ネットワーク全体に流れるトラヒックを決定する事が問題となる.
パケットの種類・量などトラヒックに関するパラメータを設定する場合,数学的に予測を行うことで決定する事も可能である.\cite{torahikku}
しかしns-3の特徴として実際のトラヒックを用いることが可能であるので,この機能を用いることでEUNETにおいては現在使用されている愛媛大学ネットワーク上のトラヒックを計測,このトラヒックを用いてシミュレーションを行うことにより,より現実に近いモデル化を行うことができる.
このシミュレーションモデルを用いて性能評価を行うことが,本研究における最終目標であると言える.
\end{comment}

% 自分の内容
 本研究は,