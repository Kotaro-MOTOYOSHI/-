%#####################################################################
\chapter{まとめ・今後の課題}
%#####################################################################

 本章では、本研究の考察及びまとめを行い、本研究における今後の課題を述べる。

本研究は、スーパーコアの改善を目的としたネットワークモデルの構築をネットワークシミュレータns-3上で行った。
当初の目標である、スーパーコアを経由したパケット転送の制御は、結果を見てみると問題なく実装されているといって良いだろう。

本論文ではOpenFlowを用いたモデル構築の初期段階しか本研究で扱うことができなかったが、今後大規模なモデル構築をする際の一助となればよいと考える。
ns-3を用いてネットワークモデルを構築するという内容に関する日本語論文およびテキスト、Webページが滅多に存在せず、更にOpenFlowを用いるという条件を加えるとその数は更に減る。
本論文により、OpenFlowを用いたシミュレーションの一例が増えることに貢献できた。
% 今後、技術が更に発展しOpenFlowが実用化に近づくことを切に願うばかりである。

本研究での今後の課題を以下に示す。

一つ目の課題として、構築したネットワークモデルと現EUNETの仕様の乖離が挙げられる。
本研究では、通信されるパケットは全てスーパーコアを経由しホストに届けられるという前提のもとで、ネットワークモデルを構築した。
しかし、現EUNETはVLANを用いてセグメントを分けるという処理を行っている。
ホスト間のセグメントが別ならば、通信されるパケットはスーパーコアを用いて検疫したのち、宛先のホストへと届けられるが、セグメントが同じならば、通常のスイッチおよびルータと同じように振る舞いパケットを通信する。
つまり、OpenFlow技術を用いてEUNETを表現しシミュレーションするためには、以上のような仕様追加を行い、再度シミュレーションをする必要がある。
これに関しては、OpenFlow技術の仕様上、比較的簡単に行うことができるが、OpenFlowのバージョンにより融通が利かない可能性もある。

二つ目の課題は、OpenFlowスイッチ、コントローラおよびホストいずれも実機を用いた実験を行っていないということがある。
2.1節で説明したように、ns-3ではTAP Bridgeを用いることによって、シミュレーションモデルの一部を実機に繋ぐことができ、より現実的な実験を行うことができる。
実際にEUNETをOpenFlowを用いて構築しようとするとき、一般に市販されているOpenFlow対応スイッチおよびコントローラを用いることがほとんどであろう。
しかし、今回の実験では市販品を用いた実験ではなく、全てns-3および別添のOpenFlowモジュール群でサポートされているモジュールを用いてスイッチおよびコントローラを作成した。
つまり、本研究で提案した手法はそのまま引き継げる可能性があるが、作成したプログラムをそのまま適用できないと考えられる。
OpenFlowのバージョンによる制約も導入するときのバージョンによって異なる可能性があるため、今回の手法が友好的でない可能性があるなど問題がある。