%#####################################################################
\chapter{OpenFlowを用いたネットワークモデルの動作検証}
%#####################################################################

 本研究は,ネットワークシミュレータ上であるns-3上でOpenFlowを用いたネットワークモデルを構築し、シミュレーションを行い、ネットワークモデルが正常に動作するかを検証することが目的である。
ns-3では、シミュレーションによるトラフィックの流れをキャプチャすることが可能であるため、これを用いて作成したネットワークモデルの動作の検証を行う。
今回の動作検証の通信プロトコルはUDPとし、シミュレーション時間の前半に正方向の通信を、後半に逆方向の通信を行うようにした。

テストシナリオにおけるシミュレーションでは、パケット通信を可視化するPyVizでの表示と、PCAPファイル形式でパケットをダンプすることが可能となっている。
本章では、主にこの2つを用いて検証結果の評価を行った。

\section{ネットワークモデルの動作検証}

動作検証を行う際に、ホストの位置関係に関する2つのケース想定し、テストシナリオを作成した。

\begin{itemize}
	\item 送信元と送信先のコアスイッチの所属が違う場合
	\item 送信元と送信先のコアスイッチの所属が同じ場合
\end{itemize}

この2つのテストシナリオが正常に動作したならば、任意の2つのホスト間の通信は正常に動作したといえる。

\subsection{送信元と送信先のコアスイッチの所属が違う場合}

送信元と送信先のコアスイッチの所属が違う場合の通信として、図のような通信経路が一例として挙げられる。
% 送信元と送信先のコアスイッチの所属が違う場合の通信として、図 \ref{fig:4-1}のような通信経路が一例として挙げられる。
この図に従い、C++を用いてテストシナリオを作成し、シミュレータ上で実行した。

% \begin{figure}[tb]
% \begin{center}
% \scalebox{0.3}{\includegraphics{重信トポロジ.eps}} 
% \caption{送信元と送信先のコアスイッチの所属が違う場合の例}
% \label{fig:4-1}
% \end{center}
% \end{figure}

\begin{comment}
\begin{figure}[tb]
\begin{center}
\begin{tabular}{c}

% 1
\begin{minipage}{0.4\hsize}
\begin{center}
\includegraphics[width=4.5cm]{./lena.eps}
\hspace{1.6cm} [1]正方向の通信
\end{center}
\end{minipage}

% 2
\begin{minipage}{0.4\hsize}
\begin{center}
\includegraphics[width=4.5cm]{./lena-affine.eps}
\hspace{1.6cm} [2]逆方向の通信
\end{center}
\end{minipage}

\end{tabular}
\caption{画像の変換例}
\label{fig:4-2}
\end{center}
\end{figure}
\end{comment}

% 図 \ref{fig:4-2}は、テストシナリオを用いてシミュレーションした際のパケット通信を可視化したものである。
図は、テストシナリオを用いてシミュレーションした際のパケット通信を可視化したものである。
図内の赤の矢印で示したノードがスーパーコアを想定したノードであり、正方向の通信、逆方向の通信ともにパケットが正常にこのノードを経由して通信を行っていることが分かる。

\begin{comment}
\begin{figure}[tb]
\begin{center}
\begin{tabular}{c}

% 1
\begin{minipage}{0.4\hsize}
\begin{center}
\includegraphics[width=4.5cm]{./lena.eps}
\hspace{1.6cm} [1]通常画像
\end{center}
\end{minipage}

% 2
\begin{minipage}{0.4\hsize}
\begin{center}
\includegraphics[width=4.5cm]{./lena-affine.eps}
\hspace{1.6cm} [2]アフィン変換(90度回転)
\end{center}
\end{minipage}

\end{tabular}
\caption{画像の変換例}
\label{fig:4-3}
\end{center}
\end{figure}
\end{comment}

% 更に、図 \ref{fig:4-3}で示すスーパーコアのそれぞれの物理ポートでのパケット出力を示すPCAPファイルによると、MACアドレスの比較によるスーパーコアの入力ポート決定の方法も正常に動作していた。
図 \ref{fig:4-3}で示すスーパーコアのそれぞれの物理ポートでのパケット出力を示すPCAPファイルによると、MACアドレスの比較によるスーパーコアの入力ポート決定の方法も正常に動作していた。
更に、重複ACKも見られなかったため、パケットロスも起こっていないことを確認した。

以上の内容からコアスイッチの所属が違う場合の通信は、すべてスーパーコアを通して通信されており、アルゴリズム通り正常に動作しているといえる。

\subsection{送信元と送信先のコアスイッチの所属が同じ場合}

送信元と送信先のコアスイッチの所属が同じ場合の通信として、図のような通信経路が一例として挙げられる。
% 送信元と送信先のコアスイッチの所属が違う場合の通信として、図 \ref{fig:4-4}のような通信経路が一例として挙げられる。
この図に従い、C++を用いてテストシナリオを作成し、シミュレータ上で実行した。

% \begin{figure}[tb]
% \begin{center}
% \scalebox{0.3}{\includegraphics{重信トポロジ.eps}} 
% \caption{送信元と送信先のコアスイッチの所属が違う場合の例}
% \label{fig:4-4}
% \end{center}
% \end{figure}

% \begin{figure}[tb]
% \begin{center}
% \scalebox{0.3}{\includegraphics{重信トポロジ.eps}} 
% \caption{エッジネットワーク(重信キャンパス)}
% \label{fig:4-5}
% \end{center}
% \end{figure}

% 図 \ref{fig:4-5}は、テストシナリオを用いてシミュレーションした際のパケット通信を可視化したものである。
図は、テストシナリオを用いてシミュレーションした際のパケット通信を可視化したものである。
図内の赤の矢印で示したノードがスーパーコアを想定したノードであり、パケットが正常にこのノードを経由して通信を行っていることが分かる。

\begin{comment}
\begin{figure}[tb]
\begin{center}
\begin{tabular}{c}

% 1
\begin{minipage}{0.4\hsize}
\begin{center}
\includegraphics[width=4.5cm]{./lena.eps}
\hspace{1.6cm} [1]通常画像
\end{center}
\end{minipage}

% 2
\begin{minipage}{0.4\hsize}
\begin{center}
\includegraphics[width=4.5cm]{./lena-affine.eps}
\hspace{1.6cm} [2]アフィン変換(90度回転)
\end{center}
\end{minipage}

\end{tabular}
\caption{画像の変換例}
\label{fig:4-3}
\end{center}
\end{figure}
\end{comment}

% 更に、図 \ref{fig:4-3}で示すスーパーコアのそれぞれの物理ポートでのパケット出力を示すPCAPファイルによると、MACアドレスの比較によるスーパーコアの入力ポート決定の方法も正常に動作していた。
図 \ref{fig:4-3}で示すスーパーコアのそれぞれの物理ポートでのパケット出力を示すPCAPファイルによると、MACアドレスの比較によるスーパーコアの入力ポート決定の方法も正常に動作していた。
更に、重複ACKも見られなかったため、パケットロスも起こっていないことを確認した。

以上の内容からコアスイッチの所属が違う場合の通信は、すべてスーパーコアを通して通信されており、アルゴリズム通り正常に動作しているといえる。