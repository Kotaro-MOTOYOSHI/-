%#####################################################################
\chapter*{概要}
%#####################################################################

% 自分の内容

 インターネットで出来る用途は幅が広く、最近ではソーシャルネットワーキングサービスやクラウドサービスなどが登場し、それに伴い、人々がインターネットに触れている時間が長くなっている。
しかし、インターネットには善良なサイトのみが存在しているわけではなく、悪質なサイトも存在しウイルスに感染してしまうなどのリスクを伴うこともある。
しかし、ユーザはウイルスへの感染を心配しているにも関わらず、情報セキュリティの脅威に関する認知、理解が十分にあるとは言えない状況が続いており、実際に情報セキュリティ被害やトラブルを経験したユーザの対処法に、「何もしなかった」と答えたユーザが一定数いたという報告もある\cite{joron1}。

上記の内容から、ウイルスがコンピュータ内に潜んでいる可能性が高く、それらのコンピュータからUSBメモリなどを通じて他のコンピュータにも感染が広がる可能性があるということは容易に想像ができる。
つまり、現在ネットワークを取り巻く脅威は、従来の外部ネットワークからの攻撃だけではなく、認知していない内部からの脅威の侵入もある。
内部からの脅威の侵入は一般的なファイアウォールなどでは防ぐことができず、ネットワーク内部に検疫システムを用意しなければならない。

本研究では、現在愛媛大学のネットワークであるEUNETを対象として、現行システムの改善を目的とした新たなシステムの提案をし、そのモデルをネットワークシミュレータ ns-3 上に構築した。
本論文では、シミュレータでネットワークを構築する際に作成したモジュール、及び採用した考え方について述べる。