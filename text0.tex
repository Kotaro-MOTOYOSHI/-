%#####################################################################
\chapter*{概要}
%#####################################################################

 インターネットで出来る用途は幅広く,最近ではソーシャルネットワーキングサービスやクラウドサービスなどが登場し,それに伴い,人々がインターネットに触れている時間が長くなっている.
しかし,インターネットには少数ではあるが悪質なサイトも存在し,ウイルス感染などのリスクを伴うこともある.
情報処理通信機構から発表されている報告書では,ウイルスおよび不正プログラムの検出数は増加傾向にあり\cite{joron2},日を追うごとに脅威の危険性は高まっている.

一方でユーザは以上のような脅威があるにも関わらず,脅威に関する認知および理解が十分とは言えない状況が続いており,実際に被害やトラブルを経験したユーザの対処法で,「何もしなかった」と答えたユーザが一定数いたという報告もある\cite{joron1}.

上記の内容から,一部のユーザのコンピュータ内に,本人の意図しないうちにウイルスが潜んでいる可能性は捨てきれず,それらのコンピュータからUSBメモリなどを通じて他のコンピュータにも感染が広がる可能性があるということは容易に想像ができる.
更にそのコンピュータが何らかのネットワークの一部であった場合,そのネットワーク全てに打撃を与える可能性もある.

つまり,現在ネットワークを取り巻く脅威の一つとして,認知していない内部からの脅威の侵入も考える必要がある.
このような内部からの脅威の侵入は,一般的なファイアウォールなど外部からの侵入を監視するシステムでは防ぐことができず,ネットワーク内部にも監視・検疫を行うシステムを用意する必要がある.

本研究では,現在愛媛大学のネットワークであるEUNETを対象として新たなシステムの提案をし,そのモデルをネットワークシミュレータ上に構築した.
本論文では,シミュレータでネットワークを構築する際に作成したモジュール,及び採用した考え方について述べる.