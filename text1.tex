%#####################################################################
\chapter{序論}
%#####################################################################

 インターネットで出来る用途は幅が広く、最近ではソーシャルネットワーキングサービスやクラウドサービスなどが登場し、それに伴い、人々がインターネットに触れている時間が長くなっている。
しかし、インターネットには善良なサイトのみが存在しているわけではなく、悪質なサイトも存在しウイルスおよび不正プログラムに感染してしまうなどのリスクを伴うこともある。
情報処理通信機構から発表された2016年の第3四半期のウイルスおよび不正アクセスに関する統計を見てみると、

しかし、ユーザはウイルスへの感染を心配しているにも関わらず、情報セキュリティの脅威に関する認知、理解が十分にあるとは言えない状況が続いており、実際に情報セキュリティ被害やトラブルを経験したユーザの対処法に、「何もしなかった」と答えたユーザが一定数いたという報告もある\cite{joron1}。

上記の内容から、ウイルスがコンピュータ内に潜んでいる可能性が高く、それらのコンピュータからUSBメモリなどを通じて他のコンピュータにも感染が広がる危険性があるということは容易に想像ができる。
つまり、現在ネットワークを取り巻く脅威は、従来の外部ネットワークからの攻撃だけではなく、認知していない内部への脅威の侵入もある。

内部への脅威の侵入が考えられる現在、主に外部との通信を監視するファイアウォールだけでは安全と言えず、ネットワーク内部にも何らかの対策システムを実装しておく必要がある。

現在の愛媛大学のネットワークであるEUNETにもスーパーコアとよばれる一連の対策システムが既に導入されている。
しかし、現在のスーパーコアのシステム、特に侵入防止システム(IPS)の周辺の構成への問題点が指摘されており、次世代のEUNETでは修正が検討されている。

本研究では、現在実用化が進んでいるOpenFlow技術を用いて、スーパーコアのシステムへのパケット転送に関する改善案を提案し、実験を行う。

実験の方法には様々あるが、今回はシミュレータを用いた仮想実験(以下、シミュレーション実験)を行う。
実際の機器を用いた実験もあり詳細なデータを獲得できるのだが、本研究の性質上、現在稼働しているEUNETのシステムの根幹に手を加えることを意味し、更に発展途上な技術を使用するため、EUNETそのものを破壊してしまう可能性も拭い切れない。

対してシミュレーション実験は、提案手法に関する実験結果がシミュレータ内で完結するため、現行システムに影響を与えることなく実験が可能となる。
更に、シミュレーション結果は数学的および論理的な演算により導出されるため、機器の劣化などによる影響を受けず、高速に結果を収集できる。

本論文では、第2章に実験で用いるシミュレータの説明し、ネットワークモデルの構築のために必要な技術について述べ、第3章でモデル構築で作成したクラスモジュールの説明を行い、第4章でモデルの動作検証の結果について述べた後、第5章で本論文のまとめおよび本研究の今後の課題について述べる。