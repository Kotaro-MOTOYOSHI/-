%#####################################################################
\chapter{序論}
%#####################################################################

 インターネットで出来る用途は幅が広く,最近ではソーシャルネットワーキングサービスやクラウドサービスなどが登場し,それに伴い,人々がインターネットに触れている時間が長くなっている.
しかし,インターネットには善良なサイトのみが存在しているわけではなく,悪質なサイトも存在しウイルスおよび不正プログラムに感染してしまうなどのリスクを伴うこともある.

情報処理通信機構から発表された2016年の第3四半期のウイルスおよび不正アクセスに関する統計によると,ウイルスおよび不正プログラムの検出件数は367,326件にのぼり,これは同年第2四半期の約2.2倍,前年第3四半期の約5.9倍と,増加傾向にある\cite{joron2}.
新たなウイルスであるとの届出件数は544件あり,前年第3四半期からの最新5四半期の間大きな差は見られない.
これは,認知されているウイルスだけでも1日に約6件もの新型ウイルスが生まれているということになる.
また,USBなど外部記憶媒体からの検出数は,最近1年間の間で3件と決して多いとは言えないものの予断を許さない.

一方で,ユーザは以上のようなウイルス感染などの脅威があるにも関わらず,認知および理解が十分にあるとは言えない状況が続いている.
情報セキュリティへの攻撃・脅威等の認知度調査\cite{joron1}によると,しばしばニュースなどで取り上げられているワンクリック詐欺やフィッシング詐欺などの認知度\footnote{この調査でいう「認知度」とは,「詳しい内容を知っている」,「概要をある程度知っている」,「名前を聞いたことがある程度」と答えた人の割合の合計値}は80\%以上と高い値を示しているが,昔から代表的な脅威の一つとして存在しているマルウェアは半数程度のユーザにしか認知されていない.
更に,最近日本でも被害が増加傾向にある,感染したコンピュータをロックし,完治させることと引き換えに「身代金」を要求するランサムウェア\cite{ransomware}に対する認知度は34.1\%とかなり低い.

さらに同調査では,実際に情報セキュリティ被害やトラブルを経験したユーザに対する対処法の質問があったが,回答率が最も高かったのは,「何もしなかった」と答えたユーザで,その割合は33.4\%と3人に1人がトラブルをそのまま放置していることが明らかになった\cite{joron1}.

上記の内容から,一部のユーザのコンピュータ内には,知らず知らずのうちにウイルスが潜んでいる可能性は捨てきれず,それらのコンピュータからUSBメモリなど外部記憶装置を通じて他のコンピュータにも感染が広がる可能性があるということは容易に想像ができる.
更にそのコンピュータが何らかのネットワークの一部であった場合,そのネットワーク全てに打撃を与える可能性もある.
つまり,現在ネットワークを取り巻く脅威は,従来の外部ネットワークからの攻撃だけではなく,認知していない内部からの脅威の侵入も考えられる.

このような内部からの脅威の侵入は,一般的なファイアウォールなど外部からの侵入を監視するシステムでは防ぐことができず,ネットワーク内部にも監視・検疫を行うシステムを用意する必要がある.

現在の愛媛大学のネットワークであるEUNETにもスーパーコアとよばれる一連の対策システムが既に導入されている.
しかし,現在のスーパーコアのシステム,特に侵入防止システム(IPS)の周辺の構成への問題点が指摘されており,次世代のEUNETでは修正が検討されている.

本研究では,現在実用化が進んでいるOpenFlow技術を用いて,スーパーコアのシステムへのパケット転送に関する改善案を提案し,動作検証のための実験を行う.

実験の方法には様々あるが,今回はシミュレータを用いた仮想実験(以下,シミュレーション実験)を行う.
実際の機器を用いた実験方法もあり,その場合は詳細なデータを獲得できるのだが,本研究の性質上,現在稼働しているEUNETのシステムの根幹に手を加える可能性があることを意味し,更に発展途上な技術を使用するため,EUNETそのものを破壊してしまいかねない.

対してシミュレーション実験は,提案手法に関する実験結果がシミュレータ内で完結するため,現行システムに影響を与えることなく実験が可能となる.
更に,シミュレーション結果は数学的および論理的な演算により導出されるため,機器の劣化などによる影響を受けず,高速に結果を収集できる.

本論文では,第2章に実験で用いるシミュレータの説明をし,ネットワークモデルの構築のために必要な技術について述べ,第3章でモデル構築で作成したクラスモジュールの説明を行い,第4章でモデルの動作検証の結果について述べた後,第5章で本論文のまとめおよび本研究の今後の課題について述べる.