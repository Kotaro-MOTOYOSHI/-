%#####################################################################
\chapter{序論}
%#####################################################################

% 森定さんの内容
\begin{comment}
 近年,インターネットの普及により,ネットワーク上を流れるトラヒックが急激に増加している.
トラヒック増加の原因としては,動画配信等のリッチコンテンツの増加,ソーシャルネットワーキングサービスなど多様なアプリケーションが次々と出現し,スマートフォンなどの新たなモバイル端末の利用拡大が進んでいることが挙げられる\cite{joron1}.

ネットワーク内部の複雑化が進むことで,ネットワーク全体の挙動理解や性能予測が困難になってきている.
ネットワークのトラヒックが爆発的に増加しているだけでなく,ネットワークの状態とユーザの振舞いが相互作用を起こし,ネットワークが予期できない状態に陥ることが報告されているからである\cite{joron1}.
さらに,この問題はネットワークが大規模であるほど顕著となる.

エンド端末間パスの物理的な最大レートである物理帯域や,背景トラヒックの影響を考慮した利用可能帯域が様々な値となり,さらに時間とともに大きく変動する事が考えられる\cite{joron9}.このように,ネットワークの振舞いを理解する事が困難になってきている現在,特に大規模ネットワークに対する性能評価手法への要求が高まっている\cite{joron2}.

ネットワークの性能評価手法として,数学的解析・シミュレーション・実験の大きく3種類の手法が存在する.

数学的解析は,対象とするネットワークを数学的にモデル化することによってその性能を解析する手法であり,振る舞いが解析的に解くことができるようなネットワークの性能評価に適している.数学的解析による大規模ネットワークの性能評価手法の研究も近年活発に行われているが \cite{joron3},モデル化する際に多くの近似が必要であるため性能評価結果は近似に依存する部分がある.

シミュレーションは,対象とするネットワークの抽象的なモデルを計算機上に構築し,ネットワークの挙動を模擬することによってその性能を評価する手法である.この手法は,数学的解析よりもより複雑なネットワークの性能評価が可能である.加えて,実際に機器を用いてネットワークを構築することなく,その性能を評価する事が可能である.
ただし,対象を完全に模擬したシミュレーションモデルを利用する事は不可能で,対象の特徴を抽出したシミュレーションモデルを利用して結果の計測を行う.
特徴を抽出する際,どのような特徴を抽出するかによりシミュレータ上での計測結果が変わってしまうことから,実験に用いたモデルの妥当性を判断する事が必要である\cite{joron4}.

一般的にネットワークシミュレータはトラヒックをパケット単位で管理するため,ネットワークの速度・規模が大きくなるにつれ,シミュレータが管理するパケット数が増加しシミュレーションに必要な計算量・メモリ量が共に増加してしまう\cite{joron5}.そのため大規模なネットワークを模擬する場合,その規模に応じた能力を持つ計算機を用意しなければならない.


実験は,実機を用いて実際にネットワークを構築することにより性能評価を行なう手法である.詳細な性能評価が可能であるが,実験では多数の計算機とネットワーク機器を用いて実験環境を構築する必要があるため,管理費や購入費に多大なコストを要するため一般には難しいとされる.加えて,機器の故障時や特殊な環境下での性能評価を行う場合,再現する事が極めて困難だという問題もある.このように,大規模ネットワークの実験は現実的ではない.しかし,愛媛大学では平成26年の9月に情報ネットワークの更新を控えており,最新かつ複雑なネットワークを実験環境として用いることができる状況にある.本研究では,シミュレーションの精度を高めるため,シミュレーションと実環境との振る舞いの差を最小化する事を目指してシミュレーション環境の構築を行う.


実際に,ネットワークシミュレータの活用はネットワークの研究や設計において利用される事がある.
ネットワークシステムを設計・開発する際,実際にネットワークを構成する前にシミュレーションを行い,起こりうる現象や効率,パフォーマンスについてシミュレーションによるネットワーク性能評価システムを用い,コンピュータネットワークシステムの性能を考慮した設計を行うことが重要であるとされる\cite{joron7}.
ネットワークの研究に関しては,各種無線環境下における特性取得が比較的容易であるため,シミュレータを用いて新規技術の検証ができ,これを用いた研究を行うことができる\cite{joron8}.

本研究では,シミュレータ上で次期愛媛大学ネットワーク(略称EUNET)を構築し性能評価を行っていくが,教育機関,企業,研究所などで使用されているネットワークシミュレータには非常に数多くのものがある.

そのため,今回はマイナー言語を使用しない,有線・無線環境をサポート,大規模ネットワーク環境をサポート,パケットキャプチャによるアニメーションを実行可能,トレース出力をサポート,オープンソースの無料シミュレータである,
という条件を全て満たすシミュレータとしてns-3を選択する.
ns-3とは,主にプロトコルや大規模システム環境の研究を行う研究者や教育者向けに開発された離散イベント型ネットワークシミュレータのことである.
詳細に関しては,2章で述べる.シミュレーションを行う際のパラメータとしては,学内を流れるトラヒックを計測して設定することが理想とされるが,
EUNETが平成26年9月導入予定であるため,本研究では現在稼働している学内ネットワークのトラヒックと,EUNETの仕様書に基づいた機器性能をパラメータとして設定し,性能評価を行う事を目的とする.

一般に,商用/フリーを問わず現存する多くの汎用ネットワークシミュレータに関して,それぞれが実装している種々のプロトコルモデルについてそのモデル化や実装詳細,基本特性といった項目が明らかにされているものは少ない.そのため,汎用の評価ツールを使用して新たにネットワークを評価する際,モデル化方法や実装内容が明らかになっていることが重要だと言える\cite{joron6}.
本研究に示される方法により,ns-3上で端末やL2スイッチ,L3スイッチの汎用的なモデル化を行い,それを用いて意図した様々なネットワークを構築し,性能評価を行うことができるようになる.
これにより,ns-3上で構築したネットワークの挙動確認や性能評価による改善点の発見を行えるだけでなく,本研究で使用した汎用的なモデルを利用し,新たなネットワークの性能評価を行う際のモデル化や,実装方法の指標とすることができる.
加えて,来年度以降は実際に構築されたEUNETの動作を検証する事でシミュレーションでの計測と実機実験としての計測を行うことができ,シミュレーションを行う際,どのようにEUNETをモデル化する事で,そのモデルが妥当性を持つかを検証する事も可能になると予想される.

本論文では,2章に各シミュレータの比較,シミュレーションを行う上で必要となるネットワーク技術,性能評価量を紹介する.3章にns-3上での大規模ネットワーク構築法と有線ネットワークを構築するために作成したモジュール,4章に作成したモジュールの動作検証について述べ,最後に5章でまとめと今後の課題を述べる.
\end{comment}

% 自分の内容
 インターネットで出来る用途は幅が広く、最近ではソーシャルネットワーキングサービスやクラウドサービスなどが登場し、それに伴い、人々がインターネットに触れている時間が長くなっている。
しかし、インターネットには善良なサイトのみが存在しているわけではなく、悪質なサイトも存在しウイルスに感染してしまうなどのリスクを伴うこともある。
しかし、ユーザはウイルスへの感染を心配しているにも関わらず、情報セキュリティの脅威に関する認知、理解が十分にあるとは言えない状況が続いており、実際に情報セキュリティ被害やトラブルを経験したユーザの対処法に、「何もしなかった」と答えたユーザが一定数いたという報告もある\cite{joron1}。

上記の内容から、ウイルスがコンピュータ内に潜んでいる可能性が高く、それらのコンピュータからUSBメモリなどを通じて他のコンピュータにも感染が広がる可能性があるということは容易に想像ができる。
つまり、現在ネットワークを取り巻く脅威は、従来の外部ネットワークからの攻撃だけではなく、認知していない内部からの脅威の侵入もある。