%#####################################################################
\chapter{関連技術}
%#####################################################################

 本研究では、提案したネットワークモデルをネットワークシミュレータを用いて構築しシミュレーションを行う。
本章では、2.1節で使用するネットワークシミュレータの説明を行い、2.2節では、ネットワークモデルを構築する際に必要となるネットワークの諸技術を説明する。

\section{ネットワークシミュレータ ns-3}

ns-3(Network Simulator Version 3)\cite{ns3} \cite{ns3text} とは、その前身であるns-2の開発に携わっている主要メンバーにより、オープンソースベース形式の離散イベント駆動型シミュレータである。
ns-3の開発チームには、Tom Henderson主任研究員を筆頭に、副主任研究員としてSumit Roy(ワシントン大学)、George Riley(ジョージア工科大学)、Sally Floyd(カリフォルニア大学バークレイ分校ISCIセンター)など、ネットワークの研究分野において著名な研究者たちが名を連ねている。
無料で使用可能なシミュレータながら、最新のサービスやプロトコルのサポートを行うなどの積極的な開発を行っているため、近年では一般的になりつつある。

ns-3は、有線でのパケット通信しか想定されていなかったなど数々の問題点があったns-2の問題点を解消し、ns-2の開発時点ではなかった新たなプロトコルの実装および、より大規模なシミュレーションの評価をより簡単にする目的で開発された。
これにより、ns-2でのシナリオ記述で採用していたObject TCL言語を廃止し、事実上C++でのデザインに統一したため、ns-3とns-2の間の互換性が無くなってしまった。
そのため、有線ネットワーク系のシミュレーションにはns-2を、無線ネットワーク系および比較的新しい技術を用いたネットワークのシミュレーションにはns-3を使うのが一般的であったが、現在では、ns-2の機能の完全移植が着実に進んでおり、徐々にns-3のシェアが高まっている。 \\

ns-3には、本研究で用いるOpenFlowに対応しているほかに、以下の機能を備える。

\subsubsection{Pythonによるシナリオファイルの作成}

ns-3は、C++言語でのデザインに統一されたが、シミュレーションするネットワークモデルの構築を行うシナリオファイルの記述には簡易スクリプト言語であるPythonも使用することができる。
Pythonを用いてシナリオファイルを作成することによって、コード量および可読性を高まり、第三者が見ても分かりやすいシナリオファイルを作成することができる。

本研究では、ns-3がOpenFlowのPythonバインディングに対応していないということと、Pythonの欠点である実行速度の遅さが如実に表れる可能性があるということから、C++用いてコーディングを行っている。

\subsubsection{tcpdumpを用いたトレースファイルの解析}

シナリオファイルにトレースファイルの生成を可能にするコードを追加することで、シナリオにある全てのノードの全ての物理ポートに対して、パケットの入力、出力のデータを持つパケットキャプチャ用ライブラリのlibpcap形式(以下、PCAP形式)のトレースファイルを生成する。
これをLinuxに標準搭載されているキャプチャツールであるtcpdumpを用いることで、PCAP形式のトレースファイルを解析することが可能である。

\subsubsection{シミュレーション結果の可視化}

ns-3では、PyVizというライブラリを用いることによって、トレースファイルを用いたシミュレーション結果のアニメーション表示が可能である。
PyVizは最新のns-3パッケージに標準搭載されており、パケットの通信経路、パケットの転送速度などを可視化して表示することができ、シミュレーション結果の時間的な流れを柔軟に変更させながら、結果の解析ができる。

\section{ネットワークモデル構築に必要なネットワーク諸技術}

\subsection{SDN}

SDNとは、Software Defined Networkの略で、ネットワーク機器をソフトウェアのように柔軟に制御できるようにという技術の総称である。\cite{SDN}

従来のネットワークは、