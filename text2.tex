%#####################################################################
\chapter{関連技術}
%#####################################################################

 本研究では、提案したネットワークモデルをネットワークシミュレータを用いて構築しシミュレーションを行う。
本章では、2.1節で使用するネットワークシミュレータの説明を行い、2.2節では本研究の柱となるOpenFlow技術の説明をし、最後に2.3節では、ネットワークモデルを構築する際に必要となるネットワークの諸技術を説明する。

\section{ネットワークシミュレータ ns-3}

ns-3(Network Simulator Version 3)\cite{ns3} \cite{ns3text} とは、その前身であるns-2の開発に携わっている主要メンバーにより、オープンソースベース形式の離散イベント駆動型シミュレータである。
ns-3の開発チームには、Tom Henderson主任研究員を筆頭に、副主任研究員としてSumit Roy(ワシントン大学)、George Riley(ジョージア工科大学)、Sally Floyd(カリフォルニア大学バークレイ分校ISCIセンター)など、ネットワークの研究分野において著名な研究者たちが名を連ねている。
無料で使用可能なシミュレータながら、最新のサービスやプロトコルのサポートを行うなどの積極的な開発を行っているため、近年では一般的になりつつある。

ns-3は、有線でのパケット通信しか想定されていなかったなど数々の問題点があったns-2の問題点を解消し、ns-2の開発時点ではなかった新たなプロトコルの実装および、より大規模なシミュレーションの評価をより簡単にする目的で開発された。
これにより、ns-2でのシナリオ記述で採用していたObject TCL言語を廃止し、事実上C++でのデザインに統一したため、ns-3とns-2の間の互換性が無くなってしまった。
そのため、有線ネットワーク系のシミュレーションにはns-2を、無線ネットワーク系および比較的新しい技術を用いたネットワークのシミュレーションにはns-3を使うのが一般的であったが、現在では、ns-2の機能の完全移植が着実に進んでおり、徐々にns-3のシェアが高まっている。 \\

ns-3には、本研究で用いるOpenFlowに対応しているほかに、以下の機能を備える。

\subsubsection{Pythonによるシナリオファイルの作成}

ns-3は、C++言語でのデザインに統一されたが、シミュレーションするネットワークモデルの構築を行うシナリオファイルの記述には簡易スクリプト言語であるPythonも使用することができる。
Pythonを用いてシナリオファイルを作成することによって、コード量および可読性を高まり、第三者が見ても分かりやすいシナリオファイルを作成することができる。

本研究では、ns-3がOpenFlowのPythonバインディングに対応していないということと、Pythonの欠点である実行速度の遅さが如実に表れる可能性があるということから、C++を用いてコーディングを行っている。

\subsubsection{実機を用いたシミュレーションをサポート}

% 「やったねたえちゃん、ns-3にはopenflowのコントローラとかIDSとかの実機をつなぐプロトコルが用意されてて、それを用いると実機を一部かませた実験ができるよ」「おいやめろ(切実)」
% 「ちなみに、この技術使えば実際流れてるトラフィックをシミュレータ上に流し込んで解析することもできるゾ」「やりますねぇ~」

ns-3には、

\subsubsection{tcpdumpを用いたトレースファイルの解析}

シナリオファイルにトレースファイルの生成を可能にするコードを追加することで、シナリオにある全てのノードの全ての物理ポートに対して、パケットの入力、出力のデータを持つパケットキャプチャ用ライブラリのlibpcap形式(以下、PCAP形式)のトレースファイルを生成する。
これをLinuxに標準搭載されているキャプチャツールであるtcpdumpを用いることで、PCAP形式のトレースファイルを解析することが可能である。

\subsubsection{シミュレーション結果の可視化}

ns-3では、PyVizというライブラリを用いることによって、トレースファイルを用いたシミュレーション結果のアニメーション表示が可能である。
PyVizは最新のns-3パッケージに標準搭載されており、パケットの通信経路、パケットの転送速度などを可視化して表示することができ、シミュレーション結果の時間的な流れを柔軟に変更させながら、結果の解析ができる。

\section{SDNとOpenFlow}

\subsection{SDN}

従来のネットワークは、スイッチおよびルータが自律的にネットワークの情報を収集し、その情報に応じてパケット通信を行ってきた\cite{openflownet}。
このとき、ネットワーク管理者が制御できる部分はあまりに限定的であったため、ネットワークユーザのニーズの多様化に対応できないなどの課題があった。
この課題を解決するために、ソフトウェアのように柔軟にネットワーク機器の制御機構を変更する技術が求められた。
このようにして考えられた技術の総称がSDN(Software Defined Network)である。

\subsection{OpenFlow}

OpenFlowとは、SDNを用いた代表的な技術で、ネットワーク機器のハードウェアレベルまでの機能の制御を行うことができる技術である\cite{openflowjapanese}。

OpenFlowを用いることで、自身で柔軟にネットワーク環境を構築し、運用する事ができる。
これにより、従来は時間をかけなければならなかったネットワーク環境の変更にも、ソフトウェアのようにコードを変更するだけで対応ができるようになり、比較的短時間で進めることができる。
更に、OpenFlow対応機器を購入し、自身でプログラムすることによって構築するため、運用コストが低いという利点がある。

\subsection{OpenFlowの仕様}

OpenFlowプロトコルには、主にスイッチをどのように振る舞わせるかを規定したもので、コントローラおよびアプリケーションを用いて定義する必要がある。


2008年12月に策定されたOpenFlow Version 0.8.9\cite{openflow} から仕様が公開された。
以下よりOpenFlowの仕様について述べる。

\subsubsection{マッチフィールド}

OpenFlowで用いるマッチフィールドを表 \ref{tab:2-1}に示す\cite{openflow}。
スイッチが予め空のフローテーブルを保有しており、入力されたパケットがフローテーブルのマッチフィールドに合致しなければ、スイッチ自身が未知のパケットと判断し、パケットの制御方法をコントローラに問い合わせる。
スイッチからコントローラに問い合わせが来たとき、コントローラ自身に記述されているアルゴリズムをもとにマッチする処理をスイッチに伝える。
コントローラから処理手順を伝えられたスイッチは、そのパケットを手順通りに処理したあと、自身のフローテーブルに追加する。

\begin{table}[tb]
	\begin{center}
		\caption{OpenFlowで用いるマッチフィールド}
		\begin{tabular}{c|c}
			\hline \hline
			フィールド & 内容 \\ \hline
			Ingress Port & 受信物理ポート \\
			Ethernet source address & 送信元MACアドレス \\
			Ethernet destination address & 宛先MACアドレス \\
			Ethernet type & パケットのプロトコルタイプ \\
			VLAN id & VLAN ID(Ethernet typeが0x8100のみ) \\
			IP source address & 送信元IPアドレス \\
			IP destination address & 宛先IPアドレス \\
			IP protocol & TCP/IPのプロトコル番号 \\
			Transport source port & 送信元のポート番号 \\
			Transport destination port & 宛先のポート番号\\ \hline
		\end{tabular}
		\label{tab:2-1}
	\end{center}
\end{table}

\subsubsection{アクション}

マッチフィールドを用いてマッチングされたパケットの処理を以下の4つから決定する\cite{openflowjapanese}。
この時、複数のアクションを指定することも可能。

\begin{itemize}
	\item 転送
	\item ヘッダの変更
	\item エンキュー
	\item 破棄
\end{itemize}

パケット転送では、入力されたパケットを指定の物理ポートへ転送する。
このとき、物理ポートを指定するほかに、OpenFlowに用意された仮想ポートを指定することができる。
主に使用する仮想ポートには、表 \ref{tab:2-2}がある。

\begin{table}[tb]
	\begin{center}
		\caption{パケット転送で使用できる仮想ポート}
		\begin{tabular}{c|c}
			\hline \hline
			仮想ポート名 & 内容 \\ \hline
			OFPP\_ALL & 全ての物理ポートへ転送 \\
			 & (入力ポートを除く) \\
			OFPP\_CONTROLLER & コントローラへ転送 \\
			OFPP\_IN\_PORT & 入力ポートへ転送 \\
			OFPP\_NORMAL & 通常のスイッチと同じ振る舞い \\
			OFPP\_FLOOD & 全ての物理ポートへ転送 \\
			 & (入力ポートおよびスパニングツリーで規制されたポート以外) \\ \hline
		\end{tabular}
		\label{tab:2-2}
	\end{center}
\end{table}

